\documentclass[review]{elsarticle}
\usepackage[nottoc]{tocbibind}
\usepackage[document]{ragged2e}
\usepackage[style=apa]{biblatex}
\addbibresource{references.bib}
\usepackage{float}
\usepackage{verbatim}
\usepackage{siunitx}
\usepackage{tcolorbox}
\usepackage{apalike}
\usepackage[utf8]{inputenc}
\usepackage{amsmath}
\usepackage{longtable}
\usepackage{amssymb}
\usepackage{pgfplots}
\usepackage{fancyhdr}
\usepackage{multicol}
\usepackage{mathtools}
\usepackage{hyperref}
\usepackage{enumitem}
\usepackage[a4paper, margin=2.5cm]{geometry}
\usepackage[english]{babel}
\usepackage{svg}
\usepackage{graphicx}
\usepackage{setspace}
\usepackage[labelfont=bf]{caption}
\pgfplotsset{width=\textwidth}
\def\n{\texttt}
\def\r{\textrm}
\def\b{\textbf}
\def\i{\textit}
\newcommand{\hs}{\hspace{5mm}}
\newcommand{\vs}{\vspace{5mm}}
\newcommand{\pc}{\textrm{ pc}}

\def\n{\texttt}
\def\r{\textrm}
\def\b{\textbf}
\def\i{\textit}
\newcommand{\prg}{\noindent}
\begin{document}
\pagestyle{fancy}
\fancyhead[R]{\today}
\fancyhead[L]{Fran Stimac -- S5177774}



\begin{center}
    \textbf{\Large Gyroscope lab -- preparation}
\end{center}
\begin{tcolorbox}
    \textbf{Q1} There are four shapes: solid sphere, spherical shell, solid cylinder, and cylindrical shell. The torque ($\tau$) applied to each shape is \SI{40}{\newton\meter}.
\end{tcolorbox}
\begin{table}[H]
\centering
\begin{tabular}{l|rr}
 & \multicolumn{1}{l}{\textbf{$\alpha$ (\unit{\radian\per\second\squared})}} & \multicolumn{1}{l}{\textbf{I (\unit{\kilogram\second\squared})}} \\ \hline
\textbf{Solid sphere} & 5 & 8.0 \\
\textbf{Spherical shell} & 3 & 13.3 \\
\textbf{Solid cylinder} & 4 & 10.0 \\
\textbf{Cylindrical shell} & 2 & 20.0
\end{tabular}
\caption{Values found using the simulation}
\label{tab:my-table}
\end{table}
\begin{itemize}
    \item Greatest angular acceleration -- Solid sphere
    \item Least angular acceleration -- Cylindrical shell
    \item Greatest moment of inertia -- Cylindrical shell
    \item Least moment of inertia -- Solid sphere
\end{itemize}
\begin{tcolorbox}
\textbf{Q2} If only the radius is halved, what happens to the magnitude of the torque?
\end{tcolorbox}
\begin{equation}
    \tau = rF\sin\theta
\end{equation}
Radius halves, therefore torque halves ($\tau \propto r$).
\begin{tcolorbox}
\textbf{Q3} If only the force is halved, what happens to the magnitude of the torque?
\end{tcolorbox}
Force halves, therefore torque halves ($\tau \propto F$).
\begin{tcolorbox}
\textbf{Q4} What is the angular acceleration of the solid cylinder?
\end{tcolorbox}
$\alpha=\SI{16}{\radian\per\second\squared}$

\begin{tcolorbox}
\textbf{Q5} What is the moment of inertia of the solid cylinder? For a solid cylinder rotating about the cylindrical axis, $I=\frac{MR^2}{2}$
\end{tcolorbox}
$I=\frac{10\times0.5^2}{2}=\SI{1.25}{\kilogram\meter\squared}$

\begin{tcolorbox}
\textbf{Q6} What is the angular velocity of the solid cylinder after 5 seconds?
\end{tcolorbox}
$\omega=16\times5=\SI{80}{\radian\per\second}$
\end{document}
